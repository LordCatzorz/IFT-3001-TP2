% !TeX root = ../main.tex
\documentclass[class=article]{standalone}

\begin{document}
\centerline{\Huge \bf Question 2}
\bigskip

\section*{Description}

Soit un menu $R$ où pour un item $x$, il y a un nombre $a_x$ d'ailes 
et $b_x$ de pintes de bière pour un coût $c_x$ d'associé.

\subsection*{Définition du tableau}

Le tableau $M$ contient le prix minimum.

\subsection*{Définition des dimensions du tableau}

La première dimension va de 0 jusqu'au nombre d'item dans le menu.

La deuxième dimension va de 0 jusqu'au nombre de d'ailes commandées.

La troisième dimension va de 0 jusqu'au nombre de bières commandées.

\subsection*{Définition d'une cellule}

La cellule $M[i,j,k]$ contient le prix minimum pour une commande de j ailes et k bières.
Elle contient l'infini si cette combinaisons de j ailes et k bières est impossible.


\subsection*{Conditions initiales}

La cellule $M[0, 0, 0] = 0$

La cellule $M[0, j, k] = \infty$ ($\forall j, k \in \Natural | j + k > 0$)

\subsection*{Récurrence}

\[
    M\crochs{i, j, k} =
    \begin{cases}
        M\crochs{i-1, j, k} & \text{si } k-b_i < 0 \\
        M\crochs{i-1, j, k} & \text{si } j-a_i < 0 \\
        \min(M\crochs{i, j-a_i, k-b_i} + c_i, M\crochs{i-1, j, k}) & \text{sinon }
    \end{cases}
\]


\section*{Analyse de la fonction commande}

Le temps d'exécution de l'algorithme dépend du nombre d'item $n$ dans le menu,
le nombre d'ailes $a$ à commandées et le nombre de pintes de bières $b$ commandées.

Cette méthode est composée de deux appels à des fonctions. Nous analyserons donc
chacune des fonctions et nous pourrons donner notre réponse selon le maximum des deux.

\subsection*{Analyse de la fonction genererTableau }

Le temps d'exécution de l'algorithme dépend du nombre d'item $n$ dans le menu,
le nombre d'ailes $a$ à commandées et le nombre de pintes de bières $b$ commandées.

L'opération de base est la comparaison \lstinline{i==0}


\end{document} 