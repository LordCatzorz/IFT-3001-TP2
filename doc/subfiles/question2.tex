% !TeX root = ../main.tex
\documentclass[class=article]{standalone}

\begin{document}
\centerline{\Huge \bf Question 2}
\bigskip

\section*{Description}

Soit un menu $R$ où pour un item $x$, il y a un nombre $a_x$ d'ailes 
et $b_x$ de pintes de bière pour un coût $c_x$ d'associé.

\subsection*{Définition du tableau}

Le tableau $M$ contient le prix minimum.

\subsection*{Définition des dimensions du tableau}

La première dimension va de 0 jusqu'au nombre d'item dans le menu.

La deuxième dimension va de 0 jusqu'au nombre de d'ailes commandées.

La troisième dimension va de 0 jusqu'au nombre de bières commandées.

\subsection*{Définition d'une cellule}

La cellule $M[i,j,k]$ contient le prix minimum pour une commande de j ailes et k bières.
Elle contient l'infini si cette combinaisons de j ailes et k bières est impossible.


\subsection*{Conditions initiales}

La cellule $M[0, 0, 0] = 0$

La cellule $M[0, j, k] = \infty$ ($\forall j, k \in \Natural | j + k > 0$)

\subsection*{Récurrence}

\[
    M\crochs{i, j, k} =
    \begin{cases}
        M\crochs{i-1, j, k} & \text{si } k-b_i < 0 \\
        M\crochs{i-1, j, k} & \text{si } j-a_i < 0 \\
        \min\pars{M\crochs{i, j-a_i, k-b_i} + c_i, M\crochs{i-1, j, k}} & \text{sinon }
    \end{cases}
\]


\section*{Analyse de la fonction commande}

Le temps d'exécution de l'algorithme dépend du nombre d'item $n$ dans le menu,
le nombre d'ailes $a$ à commandées et le nombre de pintes de bières $b$ commandées.

Cette méthode est composée de deux appels à des fonctions. Nous analyserons donc
chacune des fonctions et nous pourrons donner notre réponse selon le maximum des deux.

\subsection*{Analyse de la fonction genererTableau }

Le temps d'exécution de l'algorithme dépend du nombre d'item $n$ dans le menu,
le nombre d'ailes $a$ à commandées et le nombre de pintes de bières $b$ commandées.

Nous devons séparer l'analyse en plusieurs blocs, puisqu'il y a des appels de fonction.

\subsubsection*{Bloc A}

Le bloc A est tout ce qui se trouve à dessus des boucles \lstinline{for}.

Nous avons deux appels de fonction ici. Un appel à \lstinline{vector::size} et au constructeur
de \lstinline{Tableau::Tableau}

Ils sont tout les deux exécuté une seule fois.

L'appel à \lstinline{vector::size} ce fait en tant constant $\Theta\pars{1}$.

L'appel à \lstinline{Tableau::Tableau} ce fait en temps linéaire sur le nombre de cas du tableau.
Il n'y a pas de pire cas.
Dans notre cas, il se fait donc en tout temps à une complexité de
\begin{deriv}
\Theta\pars{(n+1)*(a+1)*(b+1)}
\<= 
\commentaire{Étendre polynôme}
\Theta\pars{nab+na+nb+ab+n+a+b+1} 
\<= 
\commentaire{Règle du maximum}
\Theta\pars{nab} 
\end{deriv}

Donc, le bloc A a une complexité de $\Theta\pars{\max\pars{1 + nab}} = \Theta\pars{nab}$


\subsubsection*{Bloc B}

Le bloc B est constitué de la boucle \lstinline{for} et de ces sous-boucles.

L'opération de base est la comparaison \lstinline{i == 0}, car c'est l'opération
exécuter le plus souvent et que tout les appels de fonction se font en temps constant,
incluant \lstinline{Tableau::at}.

Il n'y a pas de pire cas.

Le nombre de fois que cette opération peut être exécuter nous est données par 
la sommation suivante:

\begin{deriv}
    C^B\pars{n, a, b}
    \<=
    \commentaire{Définition de la sommation selon l'algorithme}
    \sum\limits_{i=0}^{n}\sum\limits_{j=0}^{a}\sum\limits_{k=0}^{b} 1
    \<=
    \commentaire{Règle de sommation}
    \sum\limits_{i=0}^{n}\sum\limits_{j=0}^{a}\pars{\pars{b-0+1} \cdot 1}
    \<=
    \commentaire{Simplification}
    \sum\limits_{i=0}^{n}\sum\limits_{j=0}^{a}\pars{b+1}
    \<=
    \commentaire{Règle de sommation}
    \sum\limits_{i=0}^{n}\pars{\pars{a - 0 + 1} \cdot \pars{b + 1}}
    \<=
    \commentaire{Simplification}
    \sum\limits_{i=0}^{n}\pars{\pars{a + 1} \cdot \pars{b + 1}}
    \<=
    \commentaire{Règle de sommation}
    \pars{n-0+1} \cdot \pars{\pars{a + 1} \cdot \pars{b + 1}}
    \<=
    \commentaire{Simplification}
    \pars{n+1} \cdot \pars{a+1} \cdot \pars{b+1}
    \<=
    \commentaire{Simplification}
    nab+na+nb+ab+n+a+b+1
    \<\in
    \commentaire{Notation aymptotique}
    \Theta\pars{nab+na+nb+ab+n+a+b+1}
    \<=
    \commentaire{Règle du maximum}
    \Theta\pars{nab}
\end{deriv}

\subsubsection*{Conclusion}

Puisque le bloc A a une complexité de $\Theta\pars{nab}$ et
le bloc B une complexité de $\Theta\pars{nab}$, la fonction genererTableau
a une complexité de $\Theta\pars{nab}$, selon la règle du maximum.

\section*{Analyse de la fonction solutionnerTableau}

\end{document} 