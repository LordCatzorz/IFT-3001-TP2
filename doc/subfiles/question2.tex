% !TeX root = ../main.tex
\documentclass[class=article]{standalone}

\begin{document}
\centerline{\Huge \bf Question 2}
\bigskip

\section*{Description}

Soit un menu $R$ où pour un item $x$, il y a un nombre $a_x$ d'ailes 
et $b_x$ de pintes de bière pour un coût $c_x$ d'associé.

\subsection*{Définition du tableau}

Le tableau $M$ contient le prix minimum.

\subsection*{Définition des dimensions du tableau}

La première dimension va de 0 jusqu'au nombre d'item dans le menu.

La deuxième dimension va de 0 jusqu'au nombre de d'ailes commandées.

La troisième dimension va de 0 jusqu'au nombre de bières commandées.

\subsection*{Définition d'une cellule}

La cellule $M[i,j,k]$ contient le prix minimum pour une commande de j ailes et k bières.
Elle contient -1 si cette combinaisons de j ailes et k bières est impossible.


\subsection*{Conditions initiales}

La cellule $M[0, j, k] = -1$

\subsection*{Récurrence}

\[
    M(i, j, k) =
    \begin{cases}
        0 & \text{si } i = j = k = 0 \\
        -1 & \text{si } i = 0 \wedge \pars{j \neq 0 \vee k \neq 0} \\
        M(i-1, j, k) & \text{si } j-a_i < 0 \\
        M(i-1, j, k) & \text{si } k-b_i < 0 \\
        M(i-1, j, k) & \text{si } M(i, j-a_i, k-b_i) == -1 \\
        M(i, j-a_i, k-b_i) + c_i & \text{si } M(i-1, j, k) == -1 \\
        \min(M(i, j-a_i, k-b_i) + c_i,& \text{sinon }    \\
            M(i-1, j, k))
    \end{cases}
\]

\end{document} 