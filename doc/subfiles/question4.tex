% !TeX root = ../main.tex
\documentclass[class=article]{standalone}

\begin{document}
\centerline{\Huge \bf Question 4}
\bigskip

\section*{Le problème est-il dans NP?}
\subsection*{Définition du certificat}

Le certificat à la forme (($G_1$, $G_2$), ($V^\prime_1$, $V^\prime_2$)) où :
\begin{itemize}
  \item $G_1$ est un graphe $\chevs{V_1, E_1}$ où $V_1$ est un vecteur de sommets et où $v_i \in V_1$ ne sont pas numérotés.
  \item $G_2$ est un graphe $\chevs{V_2, E_2}$ où $V_2$ est un vecteur de sommets et où $v_i \in V_2$ ne sont pas numérotés.
  \item $V^\prime_1$ est un vecteur d'entier de longueur $|V_1|$.
  \item $V^\prime_2$ est un vecteur d'entier de longueur $|V_2|$.
  \item $V^\prime_1$ contient des valeurs entières différentes.
  \item $V^\prime_2$ contient des valeurs entières différentes.
  \item Si $|V^\prime_1| \leq |V^\prime_2|$, alors toutes les valeurs entières 
        de $V^\prime_1$ sont aussi dans $V^\prime_2$.
  \item Si $|V^\prime_2| \leq |V^\prime_1|$, alors toutes les valeurs entières 
        de $V^\prime_2$ sont aussi dans $V^\prime_1$.
  \item Les indices d'une valeur $i$ présente à la fois dans $V^\prime_1$ et $V^\prime_2$, 
        représente les indices d'un noeud dans $V_1$ et un noeud dans $V_2$
        ayant les mêmes arrêtes dans $E_1$ et $E_2$ respectivement.
\end{itemize}

\subsection*{Pseudo-code de vérification}

\begin{lstlisting}
  Verification(g1, g2: graphe, vp1, vp2: vector<int>)
  {
    if vp1.count != g1.vertices.count
    {
      return false
    }
    
    if vp2.count != g2.vertices.count
    {
      return false
    }

    valeurG1 := new vector<int>
    valeurG2 := new vector<int>

    for i := 0 to vp1.count -1
    {
      for(j = 0 to valeurG1.count - 1)
      {
        if valeurG1[j] == vp1[i]
        {
          return false /* 2 fois meme nombre dans g1*/
        }
      }
      valeurG1.add(vp1[i])
    }

    for i := 0 to vp2.count -1 
    {
      for(j := 0 to valeurG1.count - 1)
      {
        if valeurG2[j] == vp2[i]
        {
          return false /* 2 fois meme nombre dans g2*/
        }
      }
      valeurG2.add(vp2[i])
    }

    minvp := null

    if (g1.vertices.count <= g2.vertices.count)
    {
      minvp := vp1
    }
    else
    {
      minvp := vp2
    }

    vectorIntersect := new vector<int>

    for(i := 0 to valeurG1.count -1)
    {
      for(j := 0 to valeurG2.count -1)
      {
        if (valeurG1[i] == valeurG2[j])
        {
          vectorIntersect.add(valeurG1[i])
        }
      }
    }
    
    if vectorIntersect.count != minvp.count)
    {
      return false; /* Tout les element du plus petit graphe 
                    ne sont pas present dans le plus gros */
    }
    
    for (i := 0 to minvp.count -1)
    {
      label := vp1[i];
      index2 := vp2.posof(label)
      vertex1 := g1.vertices[i]
      vertex2 := g2.vertices[index2]

      otherVertices1 = new vecor<vertex>
      for (j := 0 to vp1.edges.count -1)
      {
        if vp1.edges[j].first == vertex1
        {
          otherVertices1.add(vp1.edges[j].second)
        }
        else if vp1.edges[j].second == vertex1
        {
          otherVertices1.add(vp1.edges[j].first)
        }
      }
      
      otherVertices2 = new vector<vertex>
      for (j := 0 to vp2.edges.count -1)
      {
        if vp2.edges[j].first == vertex1
        {
          otherVertices2.add(vp2.edges[j].second)
        }
        else if vp2.edges[j].second == vertex1
        {
          otherVertices2.add(vp2.edges[j].first)
        }
      }

      if (otherVertices1.count != otherVertices2.count)
      {
        return false;
      }

      intersectCounter := 0

      for(k := 0 to otherVertices1.count -1)
      {
        for(l := 0 to otherVertices2.count -1)
        {
          if (otherVertices1[i] == otherVertices2[j])
          {
            intersectCounter := intersectCounter + 1
          }
        }
      }

      if intersectCounter != otherVertices1.count
      {
        return false;
      }
    }

    return true;
  }
\end{lstlisting}


\end{document}