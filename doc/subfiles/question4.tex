% !TeX root = ../main.tex
\documentclass[class=article]{standalone}

\begin{document}
\centerline{\Huge \bf Question 4}
\bigskip

\section*{Le problème est-il dans NP?}
\subsection*{Définition du certificat}

Le certificat à la forme (($G_1$, $G_2$), ($V^\prime_1$, $V^\prime_2$)) où :
\begin{itemize}
  \item $G_1$ est un graphe $\chevs{V_1, E_1}$ où $V_1$ est un vecteur de sommets et où $v_i \in V_1$ ne sont pas numérotés.
  \item $G_2$ est un graphe $\chevs{V_2, E_2}$ où $V_2$ est un vecteur de sommets et où $v_i \in V_2$ ne sont pas numérotés.
  \item $V^\prime_1$ est un vecteur d'entier de longueur $|V_1|$.
  \item $V^\prime_2$ est un vecteur d'entier de longueur $|V_2|$.
  \item $V^\prime_1$ contient des valeurs entières différentes.
  \item $V^\prime_2$ contient des valeurs entières différentes.
  \item $|V^\prime_1| \leq |V^\prime_2|$
  \item Toutes les valeurs entières de $V^\prime_1$ sont aussi dans $V^\prime_2$.
  \item Les indices d'une valeur $i$ présente à la fois dans $V^\prime_1$ et $V^\prime_2$, 
        représente les indices d'un noeud dans $V_1$ et un noeud dans $V_2$
        ayant les mêmes arrêtes dans $E_1$ et $E_2$ respectivement.
\end{itemize}

\subsection*{Pseudo-code de vérification}

\begin{lstlisting}
  Verification(g1, g2: graphe, vp1, vp2: vector<int>)
  {
    // bloc A
    if vp1.count != g1.vertices.count
    {
      return false
    }
    
    if vp2.count != g2.vertices.count
    {
      return false
    }
    
    if vp1.count > vp2.count
    {
      return false
    }

    valeurG1 := new vector<int>
    valeurG2 := new vector<int>
    for i := 0 to vp1.count -1
    {
      for(j = 0 to valeurG1.count - 1)
      {
        if valeurG1[j] == vp1[i]
        {
          return false /* 2 fois meme nombre dans g1*/
        }
      }
      valeurG1.add(vp1[i])
    }
    // Bloc B
    for i := 0 to vp2.count -1 
    {
      for(j := 0 to valeurG1.count - 1)
      {
        if valeurG2[j] == vp2[i]
        {
          return false /* 2 fois meme nombre dans g2*/
        }
      }
      valeurG2.add(vp2[i])
    }

    vectorIntersect := new vector<int>
    // Bloc C
    for(i := 0 to valeurG1.count -1)
    {
      for(j := 0 to valeurG2.count -1)
      {
        if (valeurG1[i] == valeurG2[j])
        {
          vectorIntersect.add(valeurG1[i])
        }
      }
    }
    
    if vectorIntersect.count != vp1.count)
    {
      return false; /* Tous les element du plus petit graphe 
                    ne sont pas present dans le plus gros */
    }
    
    // Bloc D
    for (i := 0 to vp1.count -1)
    {
      // Bloc D.A
      label := vp1[i];
      index2 := vp2.find(label)
      vertex1 := g1.vertices[i]
      vertex2 := g2.vertices[index2]
      // Bloc D.B
      otherVertices1 = new vecor<vertex>
      for (j := 0 to vp1.edges.count -1)
      {
        if vp1.edges[j].first == vertex1
        {
          otherVertices1.add(vp1.edges[j].second)
        }
        else if vp1.edges[j].second == vertex1
        {
          otherVertices1.add(vp1.edges[j].first)
        }
      }
      
      //Bloc D.C
      otherVertices2 = new vector<vertex>
      for (j := 0 to vp2.edges.count -1)
      {
        if vp2.edges[j].first == vertex1
        {
          otherVertices2.add(vp2.edges[j].second)
        }
        else if vp2.edges[j].second == vertex1
        {
          otherVertices2.add(vp2.edges[j].first)
        }
      }
      if (otherVertices1.count != otherVertices2.count)
      {
        return false;
      }

      intersectCounter := 0
      // Bloc D.D
      for(j := 0 to otherVertices1.count -1)
      {
        for(k := 0 to otherVertices2.count -1)
        {
          if (otherVertices1[k] == otherVertices2[k])
          {
            intersectCounter := intersectCounter + 1
          }
        }
      }

      if intersectCounter != otherVertices1.count
      {
        return false;
      }
    }

    return true;
  }
\end{lstlisting}

\subsection*{Analyse du pseudo-code de vérification}

La fonction de vérification prend deux graphes 
(un vecteur de noeud et un vecteur de pair de noeud(arrêt))
et deux vecteurs d'entiers.

La taille de l'instance est donc déterminée par :
\begin{itemize}
  \item $n$ : la cardinalité du vecteur d'arrêt du premier graphe;
  \item $m$ : la cardinalité du vecteur d'arrêt du deuxième graphe
  \item $x$ : la cardinalité du premier vecteur d'entier;
  \item $y$ : la cardinalité du deuxième vecteur d'entier.
\end{itemize}

Nous ne nous intéressons qu'à la borne maximale $\pars{\BigO}$ 
du comportement asymptotique du pire cas.

Nous devrons faire une analyse en plusieurs blocs.

\subsubsection*{Analyse du bloc A}

Remarquons que toutes les opérations du bloc A 
ont une complexité constante ou constante amortie.

\begin{itemize}
  \item L'ajout dans un vecteur se fait en temps constant amorti.
  \item L'accès à une valeur du vecteur se fait en temps constant.
  \item L'accès au nombre d'éléments du vecteur se fait en temps constant.
  \item La création d'un vecteur vide se fait en temps constant.
\end{itemize}

Remarquons aussi que si le vecteur du noeud du premier graphe
ne contient pas le même nombre d'éléments que le nombre d'éléments 
dans le premier vecteur d'entier, nous arrêtons le l'exécution de
la fonction.

Nous pourrons donc prendre comme opération de base
la comparaison \lstinline{valeurG1[j] == vp1[i]} qui
est l'opération exécutée le plus souvent.

La complexité peut être donnée par la sommation suivante:
\begin{deriv}
  C^A_{worst}\pars{n, m, x, y} 
  \<=
  \sum\limits_{i=0}^{x-1} \sum\limits_{j=0}^{x-1} 1
  \<=
  \sum\limits_{i=0}^{x-1} x
  \<=
  x \cdot \sum\limits_{i=0}^{x-1} 1
  \<=
  x \cdot x
  \<=
  x^2
  \<\in
  \BigO\pars{x^2}
\end{deriv}

\subsubsection*{Analyse du bloc B}

Remarquons que toutes les opérations du bloc B 
ont une complexité constante ou constante amortie.

\begin{itemize}
  \item L'ajout dans un vecteur se fait en temps constant amorti.
  \item L'accès à une valeur du vecteur se fait en temps constant.
  \item L'accès au nombre d'éléments du vecteur se fait en temps constant.
  \item La création d'un vecteur vide se fait en temps constant.
\end{itemize}

Remarquons aussi que si le vecteur du noeud du deuxième graphe
ne contient pas le même nombre d'éléments que le nombre d'éléments 
dans le deuxième vecteur d'entier, nous arrêtons le l'exécution de
la fonction.

Nous pourrons donc prendre comme opération de base
la comparaison \lstinline{valeurG2[j] == vp2[i]} qui
est l'opération exécutée le plus souvent..

La complexité peut être donnée par la sommation suivante:
\begin{deriv}
  C^B_{worst}\pars{n, m, x, y} 
  \<=
  \sum\limits_{i=0}^{y-1} \sum\limits_{j=0}^{y-1} 1
  \<=
  \sum\limits_{i=0}^{y-1} y
  \<=
  y \cdot \sum\limits_{i=0}^{y-1} 1
  \<=
  y \cdot y
  \<=
  y^2
  \<\in
  \BigO\pars{y^2}
\end{deriv}

\subsubsection*{Analyse du bloc C}

Remarquons que toutes les opérations du bloc C 
ont une complexité constante ou constante amortie.

\begin{itemize}
  \item L'ajout dans un vecteur se fait en temps constant amorti.
  \item L'accès à une valeur du vecteur se fait en temps constant.
  \item L'accès au nombre d'éléments du vecteur se fait en temps constant.
  \item La création d'un vecteur vide se fait en temps constant.
\end{itemize}

Nous pourrons donc prendre comme opération de base
la comparaison \lstinline{valeurG1[i] == valeurG2[j]} qui
est l'opération exécutée le plus souvent.

La complexité peut être donnée par la sommation suivante:

\begin{deriv}
  C^B_{worst}\pars{n, m, x, y} 
  \<=
  \sum\limits_{i=0}^{x-1} \sum\limits_{j=0}^{y-1} 1
  \<=
  \sum\limits_{i=0}^{y-1} x
  \<=
  x \cdot \sum\limits_{i=0}^{y-1} 1
  \<=
  x \cdot y
  \<\in
  \BigO\pars{x \cdot y}
\end{deriv}

\subsubsection*{Analyse du sous-bloc D.A}

Remarquons que toutes les opérations du sous-bloc D.A 
ont une complexité constante, sauf l'appel à \lstinline{find}.

\lstinline{find} à une complexité linéaire en longueur du vecteur, en pire cas.

La complexité peut donc être donnée comme suit:

$C^{D.A}_{worst}\pars{n, m, x, y} = y \in \BigO\pars{y}$

\subsubsection*{Analyse du sous-bloc D.B}

Remarquons que toutes les opérations du sous-bloc D.B 
ont une complexité constante ou constante amortie.

\begin{itemize}
  \item L'ajout dans un vecteur se fait en temps constant amorti.
  \item L'accès à une valeur du vecteur se fait en temps constant.
  \item L'accès au nombre d'éléments du vecteur se fait en temps constant.
  \item La création d'un vecteur vide se fait en temps constant.
\end{itemize}

Nous pourrons donc prendre comme opération de base
la comparaison \lstinline{vp1.edges[j].first == vertex1} qui
est l'opération exécutée le plus souvent.

La complexité peut être donnée par la sommation suivante:

\begin{deriv}
  C^{D.B}_{worst}\pars{n, m, x, y} 
  \<=
  \sum\limits_{j=0}^{n-1} 1
  \<=
  n
  \<\in
  \BigO\pars{n}
\end{deriv}

\subsubsection*{Analyse du sous-bloc D.C}

Remarquons que toutes les opérations du sous-bloc D.C 
ont une complexité constante ou constante amortie.

\begin{itemize}
  \item L'ajout dans un vecteur se fait en temps constant amorti.
  \item L'accès à une valeur du vecteur se fait en temps constant.
  \item L'accès au nombre d'éléments du vecteur se fait en temps constant.
  \item La création d'un vecteur vide se fait en temps constant.
\end{itemize}

Nous pourrons donc prendre comme opération de base
la comparaison \lstinline{vp2.edges[j].first == vertex1} qui
est l'opération exécutée le plus souvent.

La complexité peut être donnée par la sommation suivante:

\begin{deriv}
  C^{D.C}_{worst}\pars{n, m, x, y} 
  \<=
  \sum\limits_{j=0}^{m-1} 1
  \<=
  m
  \<\in
  \BigO\pars{m}
\end{deriv}

\subsubsection*{Analyse du sous-bloc D.D}

Remarquons que toutes les opérations du sous-bloc D.D 
ont une complexité constante.

\begin{itemize}
  \item L'accès à une valeur du vecteur se fait en temps constant.
  \item L'accès au nombre d'éléments du vecteur se fait en temps constant.
\end{itemize}

Nous pourrons donc prendre comme opération de base
la comparaison \lstinline{otherVertices1[i] == otherVertices2[j]} qui
est l'opération exécutée le plus souvent.

La complexité peut être donnée par la sommation suivante:

\begin{deriv}
  C^{D.D}_{worst}\pars{n, m, x, y} 
  \<=
  \sum\limits_{j=0}^{n-1} \sum\limits_{k=0}^{m-1} 1
  \<=
  \sum\limits_{j=0}^{n-1} m
  \<=
  m \cdot \sum\limits_{j=0}^{n-1} 1
  \<=
  m \cdot n
  \<\in
  \BigO\pars{m \cdot n}
\end{deriv}

\subsubsection*{Analyse du bloc D}

Le bloc D fait appel a chacun des sous-Bloc D.? de façon séquentielle.

En pire cas, il fera appel a chacun d'entre eux au complet.

Mon l'analyse asymptotique, nous pouvons donc prendre le sous-bloc 
qui a la complexité la plus élevée.

Dans notre cas, ce serait donc le sous-bloc D.D, 
car $\max{y, n, m, m \cdot n} = m  \cdot m \cdot n$ si $m > 2 ^ n > 2$.

On peut conclure que $m \cdot n >= y$ si $m > 2 ^ n > 2$, 
car il y a au moins $nbNoeud-1$ arrêt dans un graphe et y représente
le nombre de noeuds d'un graphe.

La complexité du bloc D peut donc être donnée par la sommation suivante:

\begin{deriv}
  C^{D}_{worst}\pars{n, m, x, y} 
  \<=
  \sum\limits_{i=0}^{x-1} \pars{m \cdot n}
  \<=
  \pars{m \cdot n} \cdot \sum\limits_{i=0}^{x-1} 1
  \<=
  \pars{m \cdot n} \cdot x
  \<=
  m \cdot n \cdot x
  \<\in
  \BigO\pars{m \cdot n \cdot x}
\end{deriv}


\subsubsection*{Conclusion de l'analyse}

La fonction fait appel a chacun des blocs de façon séquentielle.

En pire cas, elle fera appel à chacun.

Nous aurons donc:

\begin{deriv}
  C_{worst}\pars{n, m, x, y}
  \<=
  \max(x^2, y^2, x \cdot y, m \cdot n \cdot x)
  \<\leq
  \commentaire{$\forall n,m, x,y | x > 2 \wedge y > 2 \wedge n \geq x-1 \wedge n \geq y-1 $}
  m \cdot n \cdot x
  \<\in
  \BigO\pars{m \cdot n \cdot x}
\end{deriv}

Puisque $\BigO\pars{m \cdot n \cdot x}$ est polynomial, alors la fonction de vérification
appartient à $P$.

\subsection*{Analyse en pire cas de l'algorithme de réduction}

La fonction de réduction prend un graphe et modifie deux autres graphes pour la sortie.

La taille de l'instance est par :
\begin{itemize}
  \item $n$ : le nombre de sommets du graphe entrant;
  \item $m$ : la cardinalité du vecteur d'arrêt du graphe entrant.
\end{itemize}

Nous ne nous intéressons qu'à la borne maximale $\pars{\BigO}$ 
du comportement asymptotique du pire cas.

Nous devrons faire une analyse en plusieurs blocs.

\subsubsection*{Bloc A}

Le bloc A est composé de la boucle \lstinline{for} et de la ligne qui suit (grapheG1).

Comptons le nombre de fois que \lstinline{std::vector::emplace_back} est exécuté.

Nous pouvons prendre cette opération, car c'est elle qui a la plus grande complexité (constante amortie),
en plus d'être appelé le plus souvent.

La complexité peut donc être donnée par la sommation suivante:

\begin{deriv}
  C_{worst}^D\pars{n, m}
  \<=
  \sum\limits_{i = 0}^n 1
  \<=
  n
  \<\in
  \Theta\pars{n}
\end{deriv}

\subsubsection*{Bloc B}

Le bloc A est composé de la copie dans le grapheG2.

La copie de la structure de graphe inclut une copie d'un entier
et une copie d'un vecteur.

La copie de l'entier se fait en $\Theta\pars{1}$.

Une copie d'un vecteur est linéaire sur le nombre d'éléments. Puisqu'il y a $m$
éléments dans le vecteur d'arrêtes, la copie du vecteur aura une complexité de $\Theta\pars{m}$.

La copie du graphe est donc en $C^B_{worst}\pars{n,m}\Theta\pars{m+1} = \Theta\pars{m}$.


\subsubsection*{Conclusion de l'analyse de la réduction}

La réduction d'exécute donc en :
\begin{deriv}
  C^A_{worst}\pars{n,m}
  \<= C^A_{worst}\pars{n,m} + C^B_{worst}\pars{n,m}
  \<\in
  \Theta\pars{m + n}
\end{deriv}

\subsection*{CycleHamiltonien $=$ Oui $\Rightarrow$ Sous-Graphe $=$ Oui}
Pour qu'un graphe contienne un cycle hamiltonien, il faut qu'il y ait 
un cycle reliant tous les noeuds du graphe en ne les visitant qu'une fois.

La représentation de ce chemin représente un graphe cyclique.

Donc la comparaison avec un graphe cyclique de la réduction retournera vraie,
si ce graphe contient ce cycle hamiltonien.

\subsection*{CycleHamiltonien $=$ Non $\Rightarrow$ Sous-Graphe $=$ Non}
Pour qu'un graphe ne contienne pas un cycle hamiltonien, il faut qu'il soit 
impossible de relier tous les noeuds de ce graphe en ne les visitant qu'une fois.

La représentation d'un tel chemin serait un graphe cyclique.

Puisque ce cycle n'existe pas, la comparaison avec un graphe cyclique de la 
réduction retournera faux, si ce graphe ne contient pas de cycle hamiltonien.

\end{document}
